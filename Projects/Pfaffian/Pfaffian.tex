\documentclass[a4paper,12pt]{jlreq}

\usepackage{mylualatexstyle}



%タイトル
\title{Pfaffian}
\author{}
\date{}


\begin{document}

%タイトル
\maketitle

\begin{deftcolorbox}[title = Pfaffian]
    $2n\times 2n$の反対称行列$A(A^{\TT} = -A)$に定義される以下の量を
    Pfaffianという.
    \begin{equation}
        \Pf (A) = \sum_{M} \sgn (M) \prod_{(i,j)\in M} a_{ij}
    \end{equation}
    ここで$M$は$\qty{1,\cdots,2n}$の$n$個のペアへの分割
    \begin{equation}
        \qty{ (i_1,j_1),\cdots ,(i_n,j_n) }\quad (i_1 < \cdots < i_n,\ i_1 < j_1,\cdots , i_n < j_n)
    \end{equation}
\end{deftcolorbox}

\begin{deftcolorbox}[title = Pfaffianの他の表し方]
    Pfaffianは以下の形に変形できる.
    \begin{equation}
        \Pf (A) = \frac{1}{2^n n!} \sum_{\sigma \in S_{2n}} \sgn (\sigma) \prod_{k=1}^{n} a_{\sigma(2k-1) \sigma(2k)}
    \end{equation}
    wedge積を用いて表す方法もある. まず以下のような$W(A)$を定義する.
    \begin{equation}
        W(A) \coloneq a_{ij} e^i \wedge e^j
    \end{equation}
    これは縮約を取っていることに注意. このとき
    \begin{equation}
    \begin{aligned}
        [W(A)]^n & = \sum_{\sigma \in S_{2n}} \prod_{k=1}^{2n} a_{\sigma(2k-1) \sigma(2k)} e^{\sigma(2k-1)} \wedge e^{\sigma(2k)}\\
            & = \sum_{\sigma \in S_{2n}} \sgn (\sigma) \qty[\prod_{k=1}^{2n} a_{\sigma(2k-1) \sigma(2k)}] e^1 \wedge \cdots \wedge e^{2n}\\
            & = 2^n n! \Pf (A) e^1 \wedge \cdots \wedge e^{2n}
    \end{aligned}
    \end{equation}
\end{deftcolorbox}
\newpage
ここで, 反対称行列の行列式とPfaffianの関係について述べる.
$n$次の反対称行列$A$について, その固有方程式$f(\lambda)$は
\begin{equation}
\begin{aligned}
    f(\lambda) & = \det (\lambda I - A) = \det (\lambda I + A^{\TT})\\
    & = \det (\lambda I + A)\\
    & = (-1)^n f(-\lambda)
\end{aligned}
\end{equation}
よって$n$が$\begin{pmatrix} 偶数 \\ 奇数 \end{pmatrix}$のとき, $f(\lambda)$は$\begin{pmatrix} 偶関数 \\ 奇関数 \end{pmatrix}$となる.

$n$が奇数のときは$\det A=0$となり, $n$が偶数のときは固有値は$\lambda, -\lambda$のペアで現れる.

特に, $n$が偶数の時について, 以下の関係が成り立つ.
\begin{theotcolorbox}[title = 行列式とPfaffianの関係]
    $2n\times 2n$の反対称行列$A$について
    \begin{equation}
        \label{det_pfa}
        \det A = \qty[ \Pf (A) ]^2
    \end{equation}
\end{theotcolorbox}

\begin{proof}
    (\ref{det_pfa})は$a_{ij}$に関する有限次の多項式恒等式なので, $A$が実反対称行列の場合について示せば十分である.

    補題Aより, ある可逆行列$S$が存在し,
    \begin{equation}
        S^{\TT}AS = \begin{pmatrix}
            0 & \lambda_1 & & & \\
            -\lambda_1 & 0 & & & \\
             & & \ddots & & \\
            & & & 0 & \lambda_n \\
            & & & -\lambda_n & 0
        \end{pmatrix}
        \qty( = \Lambda )
    \end{equation}

    $\Lambda$について簡単な計算により
    \begin{align}
        \det \Lambda    & = (\lambda_1 \cdots \lambda_n)^2\\
        \Pf (\Lambda)   & = \lambda_1 \cdots \lambda_n
    \end{align}
    さらに
    \begin{equation}
        \det S^{\TT}AS = \det S^{\TT} \det A \det S = (\det S)^2 \det A
    \end{equation}
    及び補題Bより
    \begin{equation}
        \Pf (S^{\TT}AS) = (\det S) \Pf (A)
    \end{equation}
    これらから
    \begin{align}
        (\det S)^2 \det A = \det \Lambda = [\Pf (\Lambda)]^2 = (\det S)^2 [\Pf (A)]^2
    \end{align}
    $\det S \neq 0$より
    \begin{equation}
        \det A = \qty[ \Pf (A) ]^2
    \end{equation}
\end{proof}

\begin{theotcolorbox}[title = 補題A]
    $2n\times 2n$の実反対称行列$A$についてある直交行列$S$が存在し,
    \begin{equation}
        S^{\TT}AS = \begin{pmatrix}
            0 & \lambda_1 & & & \\
            -\lambda_1 & 0 & & & \\
             & & \ddots & & \\
            & & & 0 & \lambda_n \\
            & & & -\lambda_n & 0
        \end{pmatrix}
    \end{equation}
\end{theotcolorbox}

\begin{proof}
    \begin{equation}
        B \coloneq A^{\TT}A = -A^2
    \end{equation}
    とおくと, $B$は半正定値実対称行列

    Bの固有値$\mu >0$について, $\mu$に対応する固有空間$U_{\mu}$を考える.
    
    $U_{\mu}$上の実単位ベクトル$x$について
    \begin{equation}
        Bx = \mu x
    \end{equation}
    ここで以下のような$y$を定める
    \begin{equation}
        y = \frac{Ax}{\sqrt{\mu}}
    \end{equation}
    このとき
    \begin{gather}
        \norm{y}^2 = \frac{1}{\mu}x^{\TT}A^{\TT} A x = \frac{1}{\mu}x^{\TT}B x = 1\\
        B y = -\frac{1}{\sqrt{\mu}} A^3 x = \frac{1}{\sqrt{\mu}} A Bx = \sqrt{\mu} A x = \mu y
    \end{gather}
    となり, $y$も$U_{\mu}$上の実単位ベクトルである. さらに, 
    \begin{equation}
        Ay = \frac{A^2x}{\sqrt{\mu}} = -\frac{Bx}{\sqrt{\mu}} = -\sqrt{\mu} x
    \end{equation}
    となる. 以下より, $x,y$は直行する
    \begin{equation}
        x^{\TT}y = x^{\TT}Ax = - y^{\TT}x = 0
    \end{equation}
    \begin{itembox}[l]{固有空間$U_{\mu}$の正規直交基底}
    $U_{\mu}$は以下のような正規直交基底$\qty{x_{\mu 1},\cdots,x_{\mu n_{\mu}}, y_{\mu 1},\cdots,y_{\mu n_{\mu}}}$で表せる.
    \begin{equation}
        A x_{\mu i} = \sqrt{\mu} y_{\mu i},\quad A y_{\mu i} = - \sqrt{\mu} x_{\mu i}
    \end{equation}
    \end{itembox}
    これは以下のように構成できる. 

    \begin{shadebox}
    $U_{\mu}$から実単位ベクトル$x_{\mu 1}$をとり, $y = \frac{Ax_{\mu 1}}{\sqrt{\mu}}$ と定める.

    $\rightarrow \cdots $

    $\rightarrow$ $\qty{x_{\mu 1},\cdots,x_{\mu i}, y_{\mu 1},\cdots,y_{\mu i}}$と直行する実単位ベクトル$x_{\mu (i+1)}$をとり, $y = \frac{Ax_{\mu (i+1)}}{\sqrt{\mu}}$ と定める. 

    $\rightarrow \cdots $

    このとき, $i < j$について, 定義より$x_{\mu j}$は$x_{\mu i}, y_{\mu i}$と直交するので
    \begin{gather}
        x_{\mu i}^{\TT}x_{\mu j} = 0\\
        x_{\mu i}^{\TT}y_{\mu j} = x_{\mu i}^{\TT} A x_{\mu j} = - y_{\mu i}^{\TT} x_{\mu j} = 0\\
        y_{\mu i}^{\TT}x_{\mu j} = 0\\
        y_{\mu i}^{\TT}y_{\mu j} = x_{\mu i}^{\TT} B x_{\mu j} = \mu x_{\mu i}^{\TT}x_{\mu j} = 0
    \end{gather}
    つまり, $\qty{x_{\mu 1},\cdots,x_{\mu n_{\mu}}, y_{\mu 1},\cdots,y_{\mu n_{\mu}}}$は正規直交基底をなす.
    \end{shadebox}
    $B$の固有値0の固有空間$U_0$については, $U_0$に正規直交基底$\qty{z_1,\cdots,z_l}$を構成すると, 
    \begin{equation}
        B z_i = 0
    \end{equation}
    より
    \begin{equation}
        \norm{Az_i}^2 = z_i^{\TT}B z_i = 0
    \end{equation}
    より
    \begin{equation}
        A z_i = 0
    \end{equation}
    $B$はHermiteなので, 異なる固有値の固有空間は直交する.
    よって, $B$のうち非0の固有空間の正規直交基底を集めて ${x_1,\cdots,x_m, y_1,\cdots,y_m}$とし, $(2m+l=2n)$
    直交行列
    \begin{equation}
        S \coloneq \qty( x_1,y_1,\cdots x_m, y_m , z_1, \cdots ,z_l )
    \end{equation}
    を定めると
    \begin{equation}
        S^{\TT} A S = \begin{pmatrix}
            0 & \sqrt{\mu_1} & & & \\
            -\sqrt{\mu_1} & 0 & & & \\
             & & \ddots & & \\
            & & & 0 & \sqrt{\mu_n} \\
            & & & -\sqrt{\mu_n} & 0
        \end{pmatrix}
    \end{equation}
    ここで, $\mu_{2m+1},\cdots,\mu_{2n} = 0$.
\end{proof}

\begin{theotcolorbox}[title = 補題B]
    $2n\times 2n$の反対称行列$A$及び可逆行列$S$について
    \begin{equation}
        \Pf (S^{\TT}AS) = (\det S) \Pf (A)
    \end{equation}
\end{theotcolorbox}

\begin{proof}
    \begin{equation}
        \Lambda \coloneq S^{\TT}AS
    \end{equation}
    について,
    \begin{equation}
        \Lambda_{ij} = (S^{\TT})_{ik} a_{kl} S_{lj} = a_{kl} S_{ki} S_{lj}
    \end{equation}
    よって
    \begin{gather}
        \begin{aligned}
            W(\Lambda) & = \Lambda_{ij} e^i \wedge e^j = a_{kl} S_{ki} S_{lj} e^i \wedge e^j\\
            & = a_{kl} f^k \wedge f^l
        \end{aligned}
    \end{gather}
    これは$e^i$から$f^i \coloneq S_{ij}e^j$への基底変換と考えることができ, 
    \begin{equation}
        \qty[W(\Lambda)]^n = 2^n n! \Pf (A) f^1 \wedge \cdots \wedge f^{2n}
    \end{equation}
    これと
    \begin{equation}
        f^1 \wedge \cdots \wedge f^{2n} = (\det S) e^1 \wedge \cdots \wedge e^{2n}
    \end{equation}
    から, 
    \begin{equation}
        \begin{aligned}
            \qty[W(\Lambda)]^n & = 2^n n! \Pf (\Lambda) e^1 \wedge \cdots \wedge e^{2n}\\
            & = 2^n n! \Pf (A) (\det S) e^1 \wedge \cdots \wedge e^{2n}
        \end{aligned}
    \end{equation}
    よって
    \begin{equation}
        \Pf (\Lambda) = (\det S) \Pf (A)
    \end{equation}
\end{proof}

\begin{deftcolorbox}[title = 行列式のwedge積を用いた表し方]
    $n\times n$の行列$A$の行列式は
    \begin{equation}
        \det A = \sum_{\sigma \in S_n} \sgn(\sigma) \prod a_{i\sigma(i)}
    \end{equation}
    であり, wedge積を用いると
    \begin{equation}
        f^i = a_{ij} e^i
    \end{equation}
    なる基底変換をした際に
    \begin{equation}
        f^1 \wedge \cdots \wedge f^n = (\det A) e^1 \wedge \cdots \wedge e^n
    \end{equation}
    となる
\end{deftcolorbox}

\end{document}