\documentclass[a4paper,12pt]{jlreq}

\usepackage{mylualatexstyle}



%タイトル
\title{集合と位相}
\author{}
\date{}


\begin{document}

%タイトル
\maketitle

\begin{deftcolorbox}[title = 集合]
    $2n\times 2n$の反対称行列$A(A^{\TT} = -A)$に定義される以下の量を
    Pfaffianという.
    \begin{equation}
        \Pf (A) = \sum_{M} \sgn (M) \prod_{(i,j)\in M} a_{ij}
    \end{equation}
    ここで$M$は$\qty{1,\cdots,2n}$の$n$個のペアへの分割
    \begin{equation}
        \qty{ (i_1,j_1),\cdots ,(i_n,j_n) }\quad (i_1 < \cdots < i_n,\ i_1 < j_1,\cdots , i_n < j_n)
    \end{equation}
\end{deftcolorbox}

\begin{tikzpicture}[
    % スタイルの定義
    dot/.style={circle, draw=black, fill=white, inner sep=1.5pt},
    reddot/.style={circle, fill=red, inner sep=1.8pt},
    bluedot/.style={circle, fill=cyan, inner sep=1.8pt},
    path_line/.style={thick, gray!80},
    red_line/.style={ultra thick, red!70}
]

    % --- 座標の定義と描画 ---

    % 左側のメインパス(赤色の全順序部分集合を含む)
    \node[dot] (L1) at (0,0) {};
    \node[dot] (L2) at (0.3,0.7) {};
    
    % 赤い鎖(全順序部分集合)
    \node[reddot] (R1) at (0.6,1.4) {};
    \node[reddot] (R2) at (1.1,1.9) {};
    \node[reddot] (R3) at (1.7,2.3) {};
    \node[reddot] (R4) at (2.2,2.6) {};
    \node[reddot] (R5) at (2.7,2.9) {};
    \node[reddot] (R6) at (3.2,3.2) {}; % 上限

    % 赤い鎖のさらに先
    \node[dot] (L3) at (3.6,3.7) {};
    \node[dot] (L4) at (4.0,4.2) {};
    \node[bluedot] (M1) at (4.4,4.8) {}; % 極大元1

    % 枝分かれ1(赤い鎖から下へ)
    \node[dot] (B1_1) at (1.5,1.2) {};
    \node[dot] (B1_2) at (1.9,0.5) {};
    \node[dot] (B1_3) at (2.2,-0.2) {};

    % 枝分かれ2(赤い鎖の中間から右へ)
    \node[dot] (B2_1) at (3.1,2.4) {};
    \node[dot] (B2_2) at (3.5,1.7) {};
    \node[dot] (B2_3) at (3.8,1.0) {};
    \node[dot] (B2_4) at (4.2,0.3) {};

    % 枝分かれ3(右上の極大元へ)
    \node[dot] (B3_1) at (4.0,2.6) {};
    \node[dot] (B3_2) at (4.4,2.1) {};
    \node[dot] (B3_3) at (4.5,3.3) {};
    \node[dot] (B3_4) at (5.0,3.1) {};
    \node[bluedot] (M2) at (4.5,3.9) {}; % 極大元2

    % 枝分かれ4(一番右側の長い枝)
    \node[dot] (B4_1) at (5.3,2.5) {};
    \node[dot] (B4_2) at (5.7,3.0) {};
    \node[dot] (B4_3) at (6.1,3.8) {};
    \node[bluedot] (M3) at (6.3,4.5) {}; % 極大元3

    % --- 線をつなぐ ---
    \draw[path_line] (L1) -- (L2) -- (R1);
    \draw[red_line] (R1) -- (R2) -- (R3) -- (R4) -- (R5) -- (R6);
    \draw[path_line] (R6) -- (L3) -- (L4) -- (M1);
    
    \draw[path_line] (R2) -- (B1_1) -- (B1_2) -- (B1_3);
    \draw[path_line] (R4) -- (B2_1) -- (B2_2) -- (B2_3) -- (B2_4);
    
    \draw[path_line] (B2_1) -- (B3_1) -- (B3_2);
    \draw[path_line] (B3_1) -- (B3_3) -- (M2);
    \draw[path_line] (B3_3) -- (B3_4) -- (B4_1) -- (B4_2) -- (B4_3) -- (M3);

    % --- ラベルと矢印 ---
    
    % 全順序部分集合
    \node[red, align=center, font=\small] at (0.2, 2.5) {全順序\\部分集合};
    
    % 上限
    \draw[<- ,red, thick] (R6) ++(0.1,0.2) -- ++(0.5,0.5) node[left, font=\small] {上限};
    
    % 極大元
    \node[blue, font=\small, above=2pt] at (M1) {極大元};
    \node[blue, font=\small, above=2pt] at (M2) {極大元};
    \node[blue, font=\small, above=2pt] at (M3) {極大元};

\end{tikzpicture}

\end{document}