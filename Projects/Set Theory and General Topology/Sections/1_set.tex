\section{集合}

\begin{deftcolorbox}[title = 集合]
    集合とは数学的要素の集まりである.
    
    $a$が集合$A$の元であることを, 記号では$a \in A$と書く.

    $A$の部分集合全体からなる集合を$A$の冪集合といい, $2^A$で表す.
\end{deftcolorbox}

\begin{deftcolorbox}[title = 集合の濃度]
    集合$A$の濃度を$\qty|A|$と表す. 集合の濃度は以下を満たす.

    $A$が有限集合のとき, $\qty|A|$は$A$の元の個数

    無限集合$A, B$について, $A$から$B$への単射が存在するなら$\qty|A| \leq \qty|B|$

    無限集合$A, B$について, $A$から$B$への全単射が存在するなら$\qty|A| = \qty|B|$
\end{deftcolorbox}

\begin{theotcolorbox}[title = Cantorの定理]
    \begin{equation}
        \qty|2^A| > \qty|A|
    \end{equation}
\end{theotcolorbox}

\begin{deftcolorbox}[title = 集合の直積]
    $\qty{X_i}$を$I$を添字集合とする集合族とする, このとき$\qty{X_i}$の直積$\prod_i X_i$は次のように表される.

    $\prod_i X_i$ は以下の性質を満たす写像$f \colon I \longrightarrow X=\bigcup_i X_i$の集まりである.
    \begin{equation}
        \forall i \in I,\  f(i) \in X_i
    \end{equation}
    もう少し分かりやすい形で書くと, 

    \begin{equation}
        \prod_i X_i = \qty{(\cdots,x_i,\cdots) | \forall i \in I,\ x_i \in X_i}
    \end{equation}
\end{deftcolorbox}

例えば集合$A,B$の直積$A \times B$は$(a,b)\ (a \in A,\ b \in B)$という順序対からなる集合である.

\begin{deftcolorbox}[title = 集合の直和]
    $\qty{X_i}$を$I$を添字集合とする集合族とする, このとき$\qty{X_i}$の直和は以下で表される.
    \begin{equation}
        \coprod_i X_i \coloneq \bigcup_i Y_i
    \end{equation}
    ここで$Y_i$は$x\in X_i$と$i$の順序対$(x,i)$全体からなる集合である.

    直和$\coprod_i X_i$の要素は$(x,i)$と表されるが, $x$は$X=\bigcup_i X_i$から選んだ要素$x$を表し, $i$はその$x$を$\qty{X_i}$のうちどの集合$X_i$から選んだかを表す.
\end{deftcolorbox}
集合$A,B$の直和$A \oplus B$について, $A\cap B = \emptyset$の場合と$A\cap B \neq \emptyset$の場合に分けて考えよう.

(i)$A\cap B = \emptyset$の場合

$A \oplus B$の要素は$(x,i)$とおけるが, $A\cap B = \emptyset$より, $x$が定まれば$i$も定まる. 
すなわち$A \oplus B \sim A \cup B$である. このことから, $A,B$の直和を$A \cup B$で表すこともある.

$A\cap B \neq \emptyset$の場合

$A \oplus B$の要素$(x,i)$について, $x$という情報だけでは$A$からとったものか$B$からとったものかがわからないことがあるので, 添字$i$で示す必要がある.