\documentclass[a4paper,12pt]{jlreq}

\usepackage{mylualatexstyle}



%タイトル
\title{$Z_2$ Class}
\author{}
\date{}


\begin{document}

%タイトル
\maketitle

\begin{theotcolorbox}[title = 時間反転演算子$\hat{T}$]
    $\hat{K}$ を波動関数の共役をとる演算子とする. $\frac{1}{2}$スピン系で
    \begin{equation}
        \hat{T} = -i\sigma_2 \hat{K} = \begin{pmatrix} 0 & -1 \\ 1 & 0 \end{pmatrix} \hat{K}
    \end{equation}
    \begin{equation}
        \hat{T}^2 = -1
    \end{equation}
    \begin{equation}
        \bra{\hat{T} \psi} \ket{\hat{T} \phi} = \bra{\phi} \ket{\psi}
    \end{equation}
    \begin{equation}
        \bra{\psi} \ket{\hat{T} \phi} = - \bra{\phi} \ket{\hat{T} \psi}
    \end{equation}
    \begin{equation}
        \begin{aligned}
            \bra{\hat{T} \psi} \hat{T} \hat{A} \hat{T}^{-1} \ket{\hat{T} \phi} & = \bra{\hat{T} \psi} \ket{\hat{T} \hat{A} \phi}\\
            & = \bra{\hat{A} \phi} \ket{\psi} = \bra{\phi} \hat{A}^{\dagger} \ket{\psi}
        \end{aligned}
    \end{equation}
\end{theotcolorbox}

\begin{theotcolorbox}[title = 時間反転対称な系]
    \begin{equation}
        \hat{T}^{-1}\hat{H}(k)\hat{T} = \hat{H}(-k)
    \end{equation}
    $T\ket{u_n(k)}$は$\hat{H}(-k)$の固有状態である
    \begin{equation}
        \hat{H}(-k)\hat{T}\ket{u_n(k)} = \hat{T} \hat{H}(k)\ket{u_n(k)} = E_n(k) \qty( \hat{T} \ket{u_n(k)} )
    \end{equation}
\end{theotcolorbox}

\begin{deftcolorbox}[title = $w$行列]
    \begin{equation}
        \hat{T} \ket{u_{\beta}(k)} = w_{\gamma \beta} (k) \ket{u_{\gamma}(-k)}
    \end{equation}
    と定める. 
    \begin{equation}
        w_{\alpha \beta}(k) = \bra{u_{\alpha}(-k)} \hat{T} \ket{u_{\beta}(k)}
    \end{equation}
\end{deftcolorbox}

\begin{theotcolorbox}[title = $w$行列の性質]
    Unitary
    \begin{equation}
        \begin{aligned}
            \qty( w(k)^{\dagger} w(k) )_{\alpha \beta} & = w^{\ast}_{\gamma \alpha} (k) w_{\gamma \beta}(k)\\
            & = \qty( \bra{u_{\gamma}(-k)} \hat{T} \ket{u_{\alpha}(k)} )^{\ast} \qty( \bra{u_{\gamma}(-k)} \hat{T} \ket{u_{\beta}(k)} )\\
            & = \bra{\hat{T} u_{\alpha}(k)} \ket{u_{\gamma}(-k)} \bra{u_{\gamma}(-k)} \ket{\hat{T} u_{\beta}(k)}\\
            & = \bra{\hat{T} u_{\alpha}(k)} \ket{\hat{T} u_{\beta}(k)}\\
            & = \bra{u_{\beta}(k)} \ket{u_{\alpha}(k)} = \delta_{\alpha \beta}
        \end{aligned}
    \end{equation}
    $w(-k)w(k) = -1$
    \begin{equation}
        \begin{aligned}
            \hat{T}^2 \ket{u_{\alpha}(k)} & = - \ket{u_{\alpha}(k)}\\
            & = w_{\beta \alpha}(k) \hat{T} \ket{u_{\beta}(-k)}\\
            & = w_{\beta \alpha}(k) w_{\gamma \beta} (-k) \ket{u_{\gamma}(k)}\\
            よって w_{\gamma \beta}(-k) w_{\beta \alpha}(k) & = -\delta_{\alpha \gamma}
        \end{aligned}
    \end{equation}
    $w(k) = -w(-k)^{\intercal}$
    \begin{equation}
        \begin{aligned}
            w_{\alpha \beta} (k) & = \bra{u_{\alpha}(-k)} \hat{T} \ket{u_{\beta}(k)}\\
            & = - \bra{u_{\beta}(k)} \hat{T} \ket{u_{\alpha}(-k)}\\
            & = -w_{\beta \alpha}(-k)
        \end{aligned}
    \end{equation}
    TRIM $\Lambda$では$w(\Lambda)$は反対称行列
\end{theotcolorbox}

\end{document}